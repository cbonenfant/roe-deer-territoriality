\documentclass[a4paper,11pt]{article}

\usepackage{oikos}
%% Useful packages
\usepackage{amsmath}
\usepackage{graphicx}
\usepackage[colorinlistoftodos]{todonotes}
\usepackage[colorlinks=true, allcolors=blue]{hyperref}
\usepackage{textcomp}

%% new commands
\newcommand{\species}[1]{(\xmakefirstuc{\emph{#1}})}

\title{Assessing the timing of territoriality and rutting behaviour in
  roe deer from GPS data}

\author{Morellet Nicolas, Bonenfant Christophe, Miele Vincent,
  \ldots\\ \& Hewison A.J. Mark}

\affiliations{ Christophe Bonenfant, Vincent Miele and Jean-Michel
  Gaillard, Université de Lyon, Université Lyon 1, CNRS, UMR5558,
  Laboratoire de Biométrie et Biologie {\'E}volutive, Bâtiment
  G. Mendel, 43 boulevard du 11 novembre 1918, F-69622, Villeurbanne,
  France - e-mail addresses: christophe.bonenfant@univ-lyon1.fr
  (Christophe Bonenfant); miele.vincent@univ-lyon1.fr (Vincent Miele);
  jean-michel.gaillard@univ-lyon1.fr (Jean-Michel Gaillard) --
}


\begin{document}
%\doublespacing
\maketitle

\newpage
\begin{abstract}
Nice abstract here
\end{abstract}

\newpage
\section*{Introduction}

Mating and reproduction affect many facets of an individual phenotype,
changing its appearance and shape or shaping its life history and
behaviour (Darwin 1871). \todo{Ajoute un commentaire} Across species
the mating system is associated with a set of behaviours aiming at
pairing individuals with sexual partners and increasing their
reproductive success (Emlen and Oring 1977; Clutton-Brock 1989;
Wittenberger 1979). Males of polygynous species frequently engaged
into intense male-male fights, do not provide parental care to
offspring and are often larger in size than females. In monogamous
species, parents of the two sexes do care for the offspring, females
chose their mating partner carefully and males often show display
courtship behaviour (ref). Mating behaviour may also vary among
populations or individuals of the same species (Lott 19XX, Dunbar
1982). The environmental context such as population density or
resource distribution can modulate what mating behaviour is expressed
by most males (Verner and Willson 1966; Caranza et al. 1990). At the
individual level, age, social status, body size or physiological
states are tightly linked with the decision to reproduce and the type
of mating tactic to be adopted. Being closely associated to
reproductive success and fitness, mating and reproductive behaviours
are under strong selection pressures and have received a lot of
attention from ecologists. A careful understanding of the evolution of
reproductive tactics and of mating systems hence requires to assess
if, when, where and how long an individual did display
reproductive-related behaviours.

Many species may, however, be secretive, inhabiting environments with
poor visibility or present a nocturnal activity, so that
reproductive-related behaviours are difficult to observe directly or
to quantify accurately (Maher and Lott, 1995). For instance, we
currently have a huge gap of knowledge in the reproductive behaviour
of XXX and XXX because of the observation limitation imposed by the
species lifestyle and its habitat preferences. In this context,
indirect measures, like display of agonistic interactions monitored by
individual acoustic (Petrusková et al., 2016) or spatial organization
can be quantified in order to infer a territorial status
\citep{wronski_home-range_2005,corlatti_hormones_2012} or to assess
whether an individual engaged into reproduction or not. GPS systems
and activity sensors provide powerful new tools to ecologists for
remote monitoring of spatial behaviour (Cagnacci et al., 2010; Kie et
al., 2010; Kays et al., 2015). These devices provide intensive data
acquisition with which to infer behavioural patterns previously
impossible or very difficult to describe exhaustively from direct
observation, particularly in closed or remote and hard to access
habitats. For example, in females of large herbivores parturition date
and its fawn survival over the following weeks were inferred from
sudden change in movement speed (caribou \species{Rangifer tanrandus}:
ref; white-tailed deer (Odocoileus hemionus): ref; red deer (Cervus
elaphus): ref). Similarly, Silva et al. (2017) described different
lekking behaviours among little bustard males (Tetrax tetrax) from GPS
locations. In marine mammals, locations of animals is frequently used
to define foraging areas of predators such as seals or large seabirds
(). Denning dates were derived from the first date with no-movement of
black bears (). Assuming that a behaviour is associated to a change in
space use, movement or activity, this behaviour should be observable
and detected in the fine-scale locations patterns of individuals (see
Table 1).

Mating is a strong driver of changes in the behaviour of individuals
during the breeding season, including their movements and
activity. Depending on female spatial distribution, males my adopt
different mating tactics (Clutton-Brock 1989) implying specific
movement patterns. For species living at low density with widely
dispersed females such a large carnivores, roaming and greater
movements to find receptive mates during the mating period is a
behaviour that is expressed by many species (monkey, bears). Males may
also engage into the defence of territories overlapping home ranges of
one or several females. Territorial behaviour often results in an
“aggressive behaviour that occurs repeatedly as a direct response to
the location of other individuals, with associated submissive
behaviour on the part of those individuals or groups to which the
aggression is directed” (Pyke et al. 1996). In addition to these
marking and agonistic behaviours, territoriality potentially affects
the ranging behaviour and movement of males. Home range of territorial
males are smaller and with a higher density than home ranges of
floaters (Brown et al. 2000). In chimpanzees patrolling males
displayed long-distance movement steps and head-directed movements,
which opposed with the more homogenous movement patterns of lactating
females (Bates and Burns 2009). Similarly, male and females polar bear
displayed contrasting movement parameters attributed to their roaming
and patrolling behaviours, respectively. By eliciting particular
movements of males, mating related behaviours should therefore be
detectable indirectly from movements or space use of individuals.

The roe deer (\emph{Capreolus capreolus}) is a weakly polygynous large
wild herbivore that inhabits most forests of Europe (Andersen et
al. 1998). Drivers of female reproduction and the reproductive tactics
of female roe deer are relatively well known (ref) and fine-scale
mating behaviours has been documented recently such as mating
excursions (Debeffe et al. 2015). Much less is known about male mating
tactics and behaviours. The mating system of roe deer has been
described as mate defence polygyny (Vanpé et al., 2008) where males
defend a mating territory from early spring through the mid-summer
rut, when fertilization of females take place. While reports suggest
that most males become territorial during their fourth summer, some
high quality individuals may successfully mate at 2.5 years old (Vanpé
et al., 2009a). Hence, despite a relatively low level of sexual
dimorphism, patterns of spatial behaviour differ markedly between the
sexes, as males only are seasonally territorial, while females do not
actively defend their home range (but see Maublanc et al.,
2012). Territoriality patrolling (Cumming 1966), changes in HR size
and scent marking (Gosling 1982). During the rut roe deer males
actively search for females, rub antlers or chase and fight with
competitors generating an increase in the overall activity of animals.

Here, we used a large sample of GPS-monitored roe deer in two
intensively monitored populations in France and Germany to infer
territorial behaviour of male roe deer at both the individual and the
population levels. More specifically, we generated metrics of movement
and activity from GPS locations and activity sensors within the
collars and analysed the temporal patterns of these metrics in
relation to current knowledge on the seasonality of territorial
behaviour in this species (Hewison et al., 1998). In particular, we
expected to observe signals of territorial behaviour from early spring
to mid-summer i/ in males, but not females (Liberg et al., 1998), and
ii/ mostly in prime-age males, but not sub-adults or yearlings (Vanpé
et al., 2009a). While high intensity GPS data have previously been
used to infer behaviours such as parturition (e.g. DeMars et al., 2013
in caribou; Asher et al., 2014 in red deer; Severud et al., 2015 in
moose; Walsh et al., 2016 in wolves), foraging (Evans et al., 2013 in
common murres) and predation (Krofel et al., 2013 in lynx; Cristescu
et al., 2015 in bears), this is, to our knowledge, the first time that
territorial and rutting behaviours has been inferred from remote GPS
monitoring (but see Silva et al., 2017 on lekking in little bustards).

\section*{Material and Methods}
\subsection*{Study areas}

To assess the timing of territoriality and rutting behaviour in roe
deer, we used the extensive GPS monitoring studies from two sites
which feature in the EURODEER database \citep{cagnacci_partial_2011}. We
selected these study sites (Fig. 1) because they had large sample
sizes for animals of both sexes and all three age-classes (see below).

The study site in the Comminges region of south-west France (N 43° 13',
E 0° 52', 110 km$^2$, hereafter AUR) is a hilly (max. 380 m a.s.l.) mixed
landscape of open fields and small woodland patches, with two large
forests (672 and 463 ha) and numerous small forest patches (covering
19.3\% of the site). The remainder of the study site consists
essentially of meadows for livestock grazing (37.2\%), crops (31.6\%),
and hedgerows (3.6\%). The human population is present throughout the
site, in small villages and farms that are distributed along the
extensive road network. The climate is oceanic, with an average annual
temperature of 11-12° C and 800 mm precipitation. The roe deer is
hunted by stalking during summer (June-August) and drive hunts with
dogs during autumn-winter (September to February).

The study site in Germany and the Czech republic (N 49° 01', E 13° 40',
3070 km², hereafter BAV) is a central European sub-mountainous forest,
including two adjacent National Parks, the Bavarian Forest National
Park (240 km$^2$) and the Šumava National Park (640 km$^2$) in Germany and
Czech republic respectively (D:49° N). There is marked variation in
elevation between 600 and 1450 m a.s.l. and the site consists
essentially of forests of Norway spruce \species{picea abies L.}. The climate
is continental, with an average annual temperature between 3°C and
6.5° C depending on elevation, and an average annual precipitation of
between 830 and 2230 mm. The roe deer is hunted...\todo{to be
  completed by Marco}

\subsection*{Data collection}
For the AUR study site, from 2002-2013, roe deer were caught during
winter (from 16th November to 27th March) using drives of 30-100
beaters and 4 km of long-nets at one of 10 capture sites. For the BAV
study site; from 2004-2015, roe deer were caught during winter
(October–March) using wooden box traps. Each roe deer was sexed and
aged at capture (juveniles ca. 6-10 months old, yearlings ca. 18-22
months old, adults ≥ 30 months old), we recorded also its weight (with
an electronic balance to the nearest 0.1 kg). Juveniles are
distinguishable from older deer by the presence of a tricuspid third
pre-molar milk tooth (Ratcliffe and Mayle, 1992), while yearlings can
be quite reliably identified from the degree of tooth wear (Hewison et
al., 1999). Then, deer were equipped with GPS collars (mainly Lotek
3300 for AUR and Vectronic Aerospace for BAV) and released on site.

\subsection*{Data processing}
We first selected and standardised location data among the two study
sites. We removed all locations recorded during the first week of
monitoring after an animal’s capture and release because of potential
behavioural alterations following capture and handling (Morellet et
al., 2009). Because the sampling regime of GPS locations differed
among and within study sites, we selected a comparable number of
locations per unit time for each individual (Morellet et al.,
2013). We split the day into four periods of six hours ([0-6[, [6-12[,
[12-18[ and [18-0[) and retained the location that was closest to the
beginning of the interval for each six hour period (i.e. a maximum of
four locations per day, with a fix interval of approximately 6
hours). We considered six months of monitoring as the minimum to
obtain a representative estimate of ranging behaviour over the
spring-summer period and we kept only those individuals with at least
75\% (or 90\%) of the scheduled locations. The final sample comprised
67 individuals at BAV and 180 individuals at AUR (or 40 and 90)
(average of 3.6 and 3.8 locations per day for 75\% and 90\% of
scheduled locations respectively).

In order to identify temporal changes in behaviour, we generated four
metrics of movement and activity expected to index a territorial
behaviour. As territorial species are known to perform movements in
order to detect potential intruders, to patrol borders, and to
reinforce territory ownership through marking behaviour directed
towards neighbours and peripheral individuals (Owen-Smith, 1977), we
expected an increase of the general mobility and activity for
individuals delimiting a territory. Then, first to describe intensity
of mobility for each individual, we calculated the residence time
(Barraquand and Benhamou, 2008) and the movement speed of each
individual and expected respectively a decrease of the residence time
and an increase of the speed of territorial individuals so as to
define and maintain a territory. Residence time describes how animals
alternate between intensive (area-concentrated) and extensive (ranging
and relocation) search modes, which are assumed to correspond to
intra-patch and inter-patch movements, respectively. We applied this
approach using a radius of 250 m so that each GPS location is
associated with an estimate of the residence time, which corresponds
to the amount of time spent in the vicinity of any location
(Barraquand and Benhamou, 2008). To measure movement speed, we
calculated the straight-line distance between consecutive GPS
locations and divided by the time elapsed. For estimating speed, we
excluded any data where the inter-fix interval was greater than 6
hours. Second, to describe intensity of activity for each individual
we used integrated activity sensors within the GPS collars and
expected an increase of activity, as already mentioned, of territorial
individuals so as to define and maintain a territory. These sensors
measure the frequency of head movements in the horizontal (X) and
vertical (Y) planes over each five minute interval. In order to link
the activity data with the location data in a comparable time series,
we aggregated the activity information across a six hour window
centred on each theoretical GPS location (i.e. at 0, 6, 12 and
18h). For example, for a location obtained at 6h, we averaged activity
data measured from 3h to 9h separately for both the X and Y axes.

\section*{Statistical analyses}
\subsection*{At the population level}
To detect territorial and/or rutting behaviour, we compared the
movement and activity metrics of male and female juveniles, yearlings
and adults. Indeed, a priori, we expected to observe signals of
territorial behaviour in males but not females, and mostly in
prime-age males, rather than sub-adults or yearlings For the analysis,
we considered the animal’s age as it related to the rut, that is,
animals captured in winter as juveniles are yearling during the
subsequent rutting period (the second of their life), yearlings at
capture become sub-adults (third rut of their life), while adults at
capture remain adult during the subsequent rutting event.

We analysed the temporal patterns of the four metrics, i.e. residence
time in a radius of 250 m (RT250), movement speed (speed), mean head
movements on the X axis (meanX) and on the Y axis (meanY). We used
generalised additive mixed models (GAMMs) to model these four
dependant variables separately, including roe deer identity as a
random factor to control for repeated observations per individual. We
used GAMM rather than standard linear models because GAMMs capture
nonlinear temporal variation more efficiently. We used cyclic
P-splines to smooth the effect of time to account for the cyclic
pattern of the dependant variables over the year. Because day length
influences the ranging behaviour of roe deer (Borger et al., 2006b),
we arbitrarily set the winter solstice (December 22nd) as the common
starting day (Julian Date = 1) for both study sites (see also Morellet
et al., 2013). We included sex (two modalities), age class (three
modalities) and study site (two modalities) as explanatory
factors. For each dependent variable separately, we compared five
models: a model containing only the spline of the Julian date
(baseline model), a model including a sex-specific temporal effect
(i.e. a spline of Julian date for each sex), a model including both a
sex-specific and age-specific temporal effect, a model including a
sex-specific and a site-specific temporal effect, and a model
including a sex-specific, age-specific and site-specific temporal
effect. We systematically included sex in all models other than the
baseline model as there was a strong a priori reason to expect the
temporal pattern should differ between sexes (see above). In order to
improve the quality of the fit of GAMM models and then to reduce the
structure of residuals, we used the Box-Cox power transformation of
the four variables using the boxcox function in the “MASS” library
implemented in R software, version 3.3.1 (R Core Team, 2016). We used
the second order Akaike’s information criterion corrected for small
sample size (AICc; Burnham and Anderson, 1998) and Akaike weights ($w$)
to select the model with the most support among the set of five
candidate models constructed a priori. All generalized additive mixed
models were fitted using the “gamm” function in the “mgcv” library
(Wood, 2006) in R software.\todo{penses-tu qu'il soit nécessaire
  d'expliquer que nous avons augmenter le lissage} Finally, because
the inclusion of non-sedentary animals might influence the analysis,
to evaluate the robustness of our results, we eliminated all
individuals that were not classified as sedentary following the
approach described in Cagnacci et al., (2011) and then repeated the
analyses.

\subsection*{At the individual level}
To see whether it is possible to discriminate territorial and
non-territorial animals, we used an unsupervised clustering method
based on the three temporal metrics speed, meanX and meanY. We did not
consider RT250 at this stage as this metric was less informative at
population level (see Results). We performed a hierarchical
clustering, a classical technique that groups individuals together if
they present similar temporal metrics. This method requires a single
distance matrix computed on the basis of the three metrics:
consequently, for any pair of individuals, we computed the quadratic
mean of 1/ the distance between the two speed metrics and 2/ the
distance between the two activity metrics (meanX and meanY taken
together) of this pair of individuals. Since the temporal metrics can
be highly fluctuating between subsequent locations, the distance
between the metrics of two individuals can be overestimated due to
irrelevant fine scale variations. To circumvent this issue, we
previously denoised and smoothed the temporal metrics one by one with
wavelets (from an original metrics, we retained the inverse discrete
wavelet transform with the lowest frequency levels as the metrics to
be considered in the clustering approach; R package “wavethresh”
REF=tps://www.springer.com/us/book/9780387759609) and we standardized
the metrics. From the hierarchical clustering results, we deduced the
partition of the individuals into two groups of similar metrics.

As territorial status was unknown, we evaluated the composition of
these groups in terms of sex and age classes since only males are
expected to be territorial and mostly adults. As we generated two
groups of individuals (see Results), we performed generalized linear
models with a logit link function to analyse groups in relation to sex
(male versus female), study site (AUR versus BAV) and age class
(yearlings, sub-adults and adults). As we expected to observe the
territorial behaviour mostly on adult roe deer and in order to make
sure we obtained the same results over the two sites, we included also
the three-way interactions between these sex, site and age class (and
all two-way interactions) in the most complex model. Due to missing
data for the hierarchical clustering of some individuals, sample size
for this analysis (individuals with at least 75\% of the scheduled
locations) was 52 and 168 individuals for BAV and AUR,
respectively.\todo{On n'a pas le même nombre d'individus à l'issu du
  clustering} To identify the best supported model, we again used the
second order Akaike’s information criterion corrected for small sample
size.

We used the “dredge” function in the MuMIn library to generate all
sets of models (Barton, 2016).

\section*{Results}
\subsection*{At the population level}

Our overall results were robust with respect to sedentary
status. Indeed, whether we restricted the analysis to sedentary
individuals only, or analysed all available individuals, the results
were comparable (see Electronic Appendix S1). Thus, here we present
results based on the whole data set.

Residence time (RT250), movement speed (speed), mean head movements on
the X axis (meanX) and on the Y axis (meanY) all showed strong
temporal variation that differed in amplitude, timing and general
pattern among sexes, age classes and study sites (AICc weight = 1,
Table 1). In males, except for residence time, the temporal variation
of the different metrics followed a comparable pattern
(Fig. 2). However, the temporal variation of the four metrics differed
markedly between adult males and females (Fig. 2). Residence time of
females showed a pronounced peak shortly after the birth period,
whereas it was more or less constant for males. These temporal
fluctuations were generally more pronounced for adults compared to
yearlings and sub-adults (Fig. 3).

For females, movement speed followed an inverted bell curve, and was
highest during winter and lowest near the summer solstice. However,
the temporal variations in actvity (head movements on the X and Y
axes) were not very consistent between the two study sites. In
contrast, for males, both movement speed and head movements (X and Y
axes) followed a bimodal distribution, with one peak in spring (around
the 20th of April and 1st of May for AUR and BAV, respectively) and
another peak in summer (around the 1st of August and 25th of July for
AUR and BAV, respectively).

\subsection*{At the individual level}
From the hierarchical clustering performed separately for each study
site on movement speed and head movements (meanX+meanY), for each
study site we retained two clusters of individuals that showed
comparable temporal patterns (see Electronic Appendix S2). One cluster
included all individuals that had two temporal peaks for movement
speed and head movements (cluster 2), the other cluster contained
individuals that showed an inverted bell curve in movement speed (with
a higher speed during winter than during summer) and only one temporal
peak (see Electronic Appendix S2). In the next step, we thus
considered individuals with two pronounced activity peaks as those
most likely to be engaged in territorial behaviour. In the analysis of
the sex-, study site- and age- dependence of this behaviour, the best
model describing the probability of having two activity peaks
contained only the two-way interaction between age and sex (AICc
weight = 0.25, Table 2, Fig. 4). The probability of having two
activity peaks was much lower for adult females (0.016$\pm$0.016) than for
adult males (0.852$\pm$0.048).

\section*{Discussion}

\bibliography{nico}

\newpage
\begin{table}[htbp]
  \centering
  \caption{Performance of the five candidate generalized additive
    mixed models for residence time in a radius of 250 m (RT250),
    movement speed (speed), mean head movements on the X axis (meanX)
    and the Y axis (meanY), including a smoothed effect of time
    (Julian date) based on a cyclic P-spline, which differed between
    sexes, between sexes and age classes, between sexes and study
    sites, and between sexes, age classes and study sites, for
    individual roe deer with at least 75\% of scheduled locations in
    the two study sites (AUR in France and BAV in Germany), from 2002
    to 2013 in AUR and from 2004 and 2015 in BAV. The most supported
    model, based on the differences in the values for $\Delta$AICc and Akaike
    weights ($w$), is reported in bold. All models including a two-way
    interaction also include main effects. AICc is the value of the
    corrected Akaike’s information criterion and df is the number of
    degree of freedom for each model.}
  
    \begin{tabular}{rrrrrr}
      \\
      \hline  
      Proxy & Models & df    & AICc  & $\Delta$AICc & $w$ \\
      \hline
      RT250 & S(time) & 266.6 & -703224.5 & 31710.8 & <0.001 \\
            & S(time by sex) & 285   & -720677.4 & 14257.9 & <0.001 \\
            & S(time by sex, age) & 355.9 & -724606.7 & 10328.6 & <0.001 \\
            & S(time by sex, study & 321.5 & -728284.1 & 6651.2 & <0.001 \\
            & \textbf{S(time by sex, age, study} & \textbf{461.4} & \textbf{-734935.3} & \textbf{0} & \textbf{1} \\        
      \hline
      Speed & S(time) & 262.4 & 335783.2 & 8361.3 & <0.001 \\
            & S(time by sex) & 278.2 & 331056.5 & 3634.6 & <0.001 \\
            & S(time by sex, age) & 339.7 & 329990.8 & 2568.9 & <0.001 \\
            & S(time by sex, study & 314.6 & 328918.7 & 1496.9 & <0.001 \\
            & \textbf{S(time by sex, age, study} & \textbf{432.7} & \textbf{327421.8} & \textbf{0} & \textbf{1} \\
      \hline
      MeanX & S(time) & 237.8 & 1001934.3 & 24639.5 & <0.001 \\
            & S(time by sex) & 256.5 & 996264.1 & 18969.3 & <0.001 \\
            & S(time by sex, age) & 329.3 & 993102.3 & 15807.5 & <0.001 \\
            & S(time by sex, study & 293.7 & 981784.4 & 4489.6 & <0.001 \\
            & \textbf{S(time by sex, age, study} & \textbf{438.2} & \textbf{977294.8} & \textbf{0} & \textbf{1} \\       
      \hline
      MeanY & S(time) & 237.6 & 1125826.3 & 19234.9 & <0.001 \\
            & S(time by sex) & 256.3 & 1121643.1 & 15051.7 & <0.001 \\
            & S(time by sex, age) & 327.4 & 1119574.2 & 12982.8 & <0.001 \\
            & S(time by sex, study & 293.7 & 1109759.4 & 3168  & <0.001 \\
            & \textbf{S(time by sex, age, study} & \textbf{433.5} & \textbf{1106591.4} & \textbf{0} & \textbf{1} \\
      \hline 
    \end{tabular}
  \label{tab:addlabel}
  \end{table}
 
\begin{table}[htbp]
  \centering
  \caption{Summaries of the candidate generalized linear models to
    investigate the probability of having two activity peaks in
    relation to sex, age class and study site, and the three-way
    interaction between sex, age class and study site, for individual
    roe deer with at least 75\% of scheduled locations in the two
    study sites (AUR in France and BAV in Germany), from 2002 to 2013
    in AUR and from 2004 and 2015 in BAV. We only report the
    top-ranked model with a $\Delta$AICc that differed by < 5 from the most
    supported model in the table (in bold) and the null model. All
    models including a two-way interaction also include main
    effects. AICc is the value of the corrected Akaike’s information
    criterion and df is the number of degree of freedom for each
    model. The ranking of the models is based on the differences in
    the values for $\Delta$AICc and Akaike weights ($w$).}
  
    \begin{tabular}{rrrrr}
    \\
    \hline
        Models & df    & AICc  & $\Delta$AICc & $w$ \\
    \hline
    \textbf{sex x age} & \textbf{6}     & \textbf{172.1} & \textbf{0}     & \textbf{0.253} \\
    sex x age  + site x sex & 8     & 172.4 & 0.31  & 0.217 \\
    sex x age + age x site & 9     & 172.6 & 0.47  & 0.2 \\
    sex x age + site & 7     & 172.9 & 0.74  & 0.175 \\
    sex x age + age x site + sex x site & 10    & 173.1 & 0.97  & 0.156 \\
    Null  & 1     & 293.6 & 121.48 & 0 \\
    \hline
    \end{tabular}
  \label{tab:addlabel}
\end{table}

\newpage

\begin{figure} [!h]
\centering
\caption{Todo.}
\end{figure}

\begin{figure} [!h]
\centering
\caption{Temporal variations in residence time within a radius of 250 m (RT250), movement speed (speed), mean head movements on the X axis (meanX) and mean head movements on the Y axis (meanY) for adult male and female roe deer in the two study sites (AUR in France and BAV in Germany), from 2002 to 2013 in AUR and from 2004 and 2015 in BAV. The time series starts on the 22nd of December. The lines (and dotted lines) represent the separate GAMM models for each variable (with the confidence intervals) including a smoothed effect of time (Julian date) based on cyclic P-spline, which differed among study sites, sexes and age classes. Based on the literature, the light gray represents the assumed territorial period, from the 1st of April to the 1st of September, and the dark gray represents the birth period for females, from the 1st of May to the 1st of June. Finally, for males the dark gray lines represent the temporal window for the observed peaks in these metrics in spring (20th of April and 1st of May for F:43° N and D:49° N, respectively) and summer (1st of August and 25th of July for F:43° N and D:49° N, respectively).}
\end{figure}

\begin{figure} [!h]
\centering
\caption{Temporal variations in mean head movements on the Y axis for male and female roe deer among age classes in the two study sites (AUR in France and BAV in Germany), from 2002 to 2013 in AUR and from 2004 and 2015 in BAV. The time series starts on the 22nd of December. The lines (and dotted lines) represent the GAMM models (with the confidence intervals) including a smoothed effect of time (Julian date) based on a cyclic P-spline, which differed among study sites, sexes and age classes. Based on the literature, the light gray represents the assumed territorial period, from the 1st of April to the 1st of September, and the dark gray represents the birth period for females, from the 1st of May to the 1st of June. Finally, for males the dark gray lines represent the temporal window for the observed peaks in these metrics in spring (20th of April and 1st of May for F:43° N and D:49° N, respectively) and summer (1st of August and 25th of July for F:43° N and D:49° N, respectively).}
\end{figure}

\begin{figure} [!h]
\centering
\caption{The probability of having two temporal peaks of activity in relation to the sex and the age class, for roe deer monitored in two study sites (AUR in France and BAV in Germany), from 2002 to 2013 in AUR and from 2004 and 2015 in BAV. The dots (and segment lines) represent the GLM models (with the confidence intervals).}
\end{figure}

\end{document}